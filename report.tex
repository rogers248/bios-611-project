\documentclass{article}
\usepackage[utf8]{inputenc}
\usepackage{indentfirst}
\usepackage{graphicx}
\usepackage{amsmath}
\usepackage{hyperref}

\setlength{\parindent}{2em}

\title{report}
\author{Tianyi Wang \\ e-mail: rogers24@ad.unc.edu}
\date{November 2021}

\begin{document}

\maketitle

\section{Introduction}

Powerlifting is a strength sport that consists of three attempts at maximal weight on three lifts: squat, bench press, and deadlift.This sport is getting more and more popular around the world. So I decided to start this project to get deeper understanding about powerlifting. The dataset is a snapshot of the OpenPowerlifting database as of 2015 to 2019. OpenPowerlifting is an organization which tracks meets and competitor results in the sport of powerlifting, in which competitors complete to lift the most weight for their class in three separate weightlifting categories. I'm using the data from it to make some data analysis in the powerlifting field recently.

\section{Data Source}

The dataset with variable descriptions and more context can be found \href{https://www.kaggle.com/open-powerlifting/powerlifting-database}.

\section{Data Visualization and Analysis}

\subsection{Country and Meets}

\includegraphics[width=0.5\textwidth]{Country and Meet.png}

We can see from the picture that in our chosen years, USA,Norway,Canada,Australia and New Zealand hold powerlifting meets a lot. And USA, as the origin country of the powerlifting sports, definitely hold meets most, 3894 times. 


\subsection{Weight Class for Men and Women}

\includegraphics[width=0.5\textwidth]{Men Weight Class.png}\\
\includegraphics[width=0.5\textwidth]{Women Weight Class.png}

From the barplot for men, the top 3 common weight class are 100kg,90kg and 110kg, which means weight class around 100kg is more popular for powerlifting. From the barplot for women, the top 3 common weight class are 67.5kg, 75kg and 60kg, which means class around 67.5kg is more popular among women. 
If we take consideration of average weight in US, the average weight of American men in 2015-16 was 89kg; for women, it was 77kg. So most of the male powerlifters are heavier than average while most female powerlifters keep ordinary weights.


\subsection{Wilk: Highest Wilk for Men and Women}

\includegraphics[width=0.5\textwidth]{Highest Wilk for Men.png}\\
\includegraphics[width=0.5\textwidth]{Highest Wilk for Women.png}

The Wilks coefficient or Wilks formula is a mathematical coefficient that can be used to measure the relative strengths of powerlifters despite the different weight classes of the lifters. It's one of the most important reference for ranking in powerlifting meets. From the wilk ranking barplots for men and women, the best wilk for men is 609 and the best wilk for women is 621. And in the data range we chose, we have 4 female powerlifters' wilk over 600 and 2 male powerlifters' wilk over 600.
Though we know most of the situations men and lift much heavier than women at the same weight class, the result shows that women have better wilk coefficient.

\subsection{Relationships between lifted weights and body weights}

\includegraphics[width=0.5\textwidth]{TotalKg.png}\\
\includegraphics[width=0.5\textwidth]{BestBenchPress.png}\\
\includegraphics[width=0.5\textwidth]{BestDeadlift.png}\\
\includegraphics[width=0.5\textwidth]{BestSquat.png}

Not like our instinct, the heavier body, the better total weights for powerlifting. We can see from the Body Weight VS Total Weight plot that body weight around 150kg performs better. Same thing happens for the three separate program, not the biggest weight has biggest strength. From the three separate plots, we can see squat weight has the most significant relationship with body weight, while the deadlift has the least.



\section{Analysis Summary}

From our analysis above, we can see USA holds powerlifting meets most, which goes far ahead of other countries. As for weight class of men and women, weight class around 100kg is more common among men and weight class around 67.5kg is more common among women. As for wilk coefficient, women have better wilk coefficients than men. And when we are talking about relationships between lifted weights and body weights, we can see bigger body weights cannot lead to heavier total lifted weights directly and body weight matters more with squat weight and less with deadlift weight.

\end{document}
